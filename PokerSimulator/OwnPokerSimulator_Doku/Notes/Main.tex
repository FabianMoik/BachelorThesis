% **************************************************************************************************
% ** SPSC Report and Thesis Template
% **************************************************************************************************
%
% ***** Authors *****
% Daniel Arnitz, Paul Meissner, Stefan Petrik
% Signal Processing and Speech Communication Laboratory (SPSC)
% Graz University of Technology (TU Graz), Austria
%
% ***** Changelog *****
% 0.1   2010-01-25   extracted from report template by Daniel Arnitz (not ready yet)
% 0.2   2010-02-08   added thesis titlepage and modified layout (not ready yet)
% 0.3   2010-02-18   added TUG logo and statutory declaration
% 0.4   2010-02-18   moved the information fields below \input{./base/packages} (encoding...)
% 0.5   2010-03-02   added \ShortTitle to fix problems with long thesis titles
%                    added \ThesisType (makes the template suitable for MSc, BSc, PhD, ... Thesis)
% 0.6   2010-06-05   added pagestyle and pagenumbering after frontmatter, packages has now type
% 0.7   2010-09      \Advisors -> \Assessors, inserted frontmatter for thesis
% 0.8   2010-11      added examples
% 0.9   2011-04      \Twosided now {true,false}, scrbook for thesis (\front-, \main-, \backmatter)
%                    added \SpecialNote for titlepage (funding, etc.), added type "homework"
% 0.10  2011-10-18   fixed two typos in \bibliographystyle{} (bug reported by Michael Tauch)
% 0.11  2011-11-09   fixed/modified preamble (bug reported by Michael Tauch)
% 0.12  2012-07-20   added ./base/opt_macros to deal with optional macros
% 0.13  2012-07-27   added \PaperSize
%
% ***** Todo *****
% - Introduction/Usage
% - explain/show preamble (with \thispagestyle, etc)
% - why doesn't \pagestyle work in preamble while \thispagestyle does? (reported by Markus Fr�hle)
% **************************************************************************************************

% **************************************************************************************************
% basic setup

\newcommand{\DocumentType}{homework} % "thesis" / "report" / "homework"
\newcommand{\DocumentLanguage}{de} % "en" / "de"
\newcommand{\PaperSize}{a4paper} % "a4paper" / "letterpaper"
\newcommand{\Twosided}{false} % "true" / "false"

% **************************************************************************************************
% template setup -- do not change these unless you know what you are doing!
\input{./base/packages_\DocumentType}
\input{./base/layout_\DocumentType}
\input{./base/macros}
% \graphicspath{{./drawings/}{./plots/}{./images/}}
% **************************************************************************************************
% ATTENTION: There is a stylesheet provided for makeindex; set makeindex to -s "./base/index.sty"
% **************************************************************************************************

% uncomment to get watermarks:
% \usepackage[first,bottom,light,draft]{draftcopy}
% \draftcopyName{ENTWURF}{160}


% **************************************************************************************************
% information fields

% general
\newcommand{\DocumentTitle}{General Information}
\newcommand{\DocumentSubtitle}{}
\newcommand{\ShortTitle}{Study Material} % used in headers (keep short!)
\newcommand{\DocumentAuthor}{Fabian Moik}
\newcommand{\DocumentDate}{Graz, \today}
%    for thesis only (will be ignored for reports)
\newcommand{\ThesisType}{Bachelor Thesis}
%\newcommand{\Organizations}{Signal Processing and Speech Communications Laboratory \\ Graz University of Technology, Austria \\[1cm] in co-operation with \\ A Nice Company \\ Cartagena, Spain} % SPSC \\ TUG \\[1cm] in cooperation with \\ A Nice Company
%\newcommand{\Supervisors}{Assoc.Prof. Dipl.-Ing. Dr. Klaus Witrisal \\ Dipl.-Ing. Paul Meissner} % Supervisor 1 \\ Supervisor 2 \\ ...
%\newcommand{\Assessors}{Univ.-Prof. Dipl.-Ing. Dr.techn. Gernot Kubin \\ Assoc.Prof. Dipl.-Ing. Dr. James J. Tobe Defined}
%\newcommand{\SpecialNote}{This work was funded by the Austrian Research Promotion Agency (FFG) under grant 123456.}
%   for report only: revision number
\newcommand{\RevPrefix}{alpha~}
\newcommand{\RevLarge}{1}
\newcommand{\RevSmall}{0}

% confidential? (can of course also be used for other messages/notes)
\newcommand{\ConfidNote}{\today}


% **************************************************************************************************
% miscellaneous

% correct bad hyphenation
\hyphenation{}
\usepackage{tabularx}
\usepackage{amsmath}
\usepackage{grffile}
\usepackage{float}
\usepackage{blindtext, graphicx}
\usepackage[labelfont=bf]{caption}
\usepackage{chngcntr}
\usepackage{mathtools}
\usepackage{hyperref}
\usepackage{listings}
\usepackage{color}

\definecolor{dkgreen}{rgb}{0,0.6,0}
\definecolor{gray}{rgb}{0.5,0.5,0.5}
\definecolor{mauve}{rgb}{0.58,0,0.82}
%\lstset{language=[Objective]C, breakindent=40pt, breaklines}

\lstset{ %
  language=C++,                  % the language of the code
  basicstyle=\footnotesize,       % the size of the fonts that are used for the code
  numbers=left,                   % where to put the line-numbers
  numberstyle=\tiny\color{gray},  % the style that is used for the line-numbers
  stepnumber=1,                   % the step between two line-numbers. If it's 1, each line 
                                  % will be numbered
  numbersep=5pt,                  % how far the line-numbers are from the code
  backgroundcolor=\color{white},  % choose the background color. You must add \usepackage{color}
  showspaces=false,               % show spaces adding particular underscores
  showstringspaces=false,         % underline spaces within strings
  showtabs=false,                 % show tabs within strings adding particular underscores
  frame=single,                   % adds a frame around the code
  rulecolor=\color{black},        % if not set, the frame-color may be changed on line-breaks within not-black text (e.g. commens (green here))
  tabsize=4,                      % sets default tabsize to 2 spaces
  captionpos=b,                   % sets the caption-position to bottom
  breaklines=true,                % sets automatic line breaking
  breakatwhitespace=false,        % sets if automatic breaks should only happen at whitespace
  title=\lstname,                 % show the filename of files included with \lstinputlisting;
                                  % also try caption instead of title
  keywordstyle=\color{blue},          % keyword style
  commentstyle=\color{dkgreen},       % comment style
  stringstyle=\color{mauve},         % string literal style
  escapeinside={\%*}{*)},            % if you want to add a comment within your code
  morekeywords={*,...}               % if you want to add more keywords to the set
}
% switches
\newboolean{OptDraftMode}
\newboolean{DisplayContentBoxes}
% \setboolean{OptDraftMode}{true} % optional draft mode for pixel graphics (speed up generation; add \OptDraft to options)
% \setboolean{DisplayContentBoxes}{true} % optional boxes with contents (\ContentBox{Content}{NumPages} can be used as "sticky note" with planned contents)
%   load
\input{./base/opt_macros}

\renewcommand*{\thesection}{\arabic{section}}
\newcommand*{\xchapter}{\setcounter{section}{0}\addchap}
% **************************************************************************************************
% **************************************************************************************************
%
 %**************************************************************************************************
\begin{document}
%%%%%%%%% begin snippet
%% You need to add the package "tabularx".
%% Place the snippet right after \begin{document}

% need tabularx
%\usepackage{tabularx}

\begin{titlepage}
       \begin{center}
             \begin{huge}
				   %% Update assignment number here
                   \textbf{Bachelor Thesis - Poker Simulator}
             \end{huge}
       \end{center}
       \begin{center}
             \begin{large}
		Study Material
             \end{large}
       \end{center}
       \begin{center}
             \begin{large}
                 \textbf{Fabian Moik}
             \end{large}
       \end{center}
\end{titlepage}

%%%%%%%%% end snippet
% **************************************************************************************************
% titlepage
%\input{./base/titlepage_\DocumentType}\emptydoublepage

% for thesis: switch to frontmatter
%\ifthenelse{\equal{\DocumentType}{thesis}}{\pagestyle{empty}\pagenumbering{roman}}{}


% **************************************************************************************************
% **************************************************************************************************
% user-defined part

% FOR THESIS: ADD THE PREAMBLE (ABSTRACT, KURZFASSUNG, ...) HERE (also add an \emptydoublepage in between), e.g.:
%    \input{my-abstract}
%    \emptydoublepage
%    \input{my-kurzfassung}
%    \emptydoublepage
%    ...
% FEEL FREE TO USE \emptypage AND \emptydoublepage TO ADJUST THE LAYOUT
% USE \thispagestyle{empty} for abstract, etc.

% for thesis: statutory declaration
\ifthenelse{\equal{\DocumentType}{thesis}}{\input{./base/declaration}}{}

% TOC
%\emptydoublepage
\tableofcontents

% for thesis: make sure we switch back to standard pagestyles/numbering
\ifthenelse{\equal{\DocumentType}{thesis}}{\emptydoublepage\pagestyle{scrheadings}\pagenumbering{arabic}\mainmatter}

% FOR THESIS: YOU CAN SET THE PAGECOUNTER HERE TO MAKE IT IDENTICAL TO THE PDF PAGE NUMBER
\ifthenelse{\equal{\DocumentType}{thesis}}{\setcounter{page}{7}}{}



%%%%%%%%%%%%%%%%%%%%%%%%%%%%%%%%%%%%%%%%%%%%%%%%%%%%%%%%%%%%%%%%%

% **************************************************************************************************
% mainmatter
\newpage
% %%%%%%%%%%%%%%%%%%%%% 	1	 %%%%%%%%%%%%%%%%%%%%%%%%%
%    \emptydoublepage %FOR THESIS: ALWAYS START CHAPTERS AT RIGHT SIDE
\counterwithin{figure}{section}
\counterwithin{section}{chapter}

%%%%%%%%%%%%%%%%%% What I want to write %%%%%%%%%%%%%%%%%%%%
\chapter{Poker Simulator}
\section{Basic Structure}
\subsection{Currently working on}
\begin{itemize}
\item lets players make moves and check when a betting round has finished
\subitem TODO: think of all scenarios where different people are the last raiser (bigblind, small blind, no one, did some go all-in? etc...)
\item next check for action to be valid
\item the \textbf{Action} class should check if the action is valid -> see rules for raising
\subitem \textbf{test the isValidAction} method for all scenarios 
\item create an \textbf{Information class} which holds all the information that is accessible to an AI.
\item reseating players if needed, to balance game
\item make a concept of what you need for your neural network and how the training is done.
\subitem the training is purely done by evolving the weights and features there for no backprop algorithm or gradient calculation is needed
\end{itemize}
%%%%%%%%%%%%%%
\subsection{Ideas and Implementation Hierarchy}
\begin{itemize}
\item A \textbf{game} is run simultaneously  on \textbf{x tables}
\subitem each \textbf{table} holds \textbf{y players}, a \textbf{deck of cards} and a \textbf{dealer (observer)}
\item the \textbf{dealer} deals the cards, and holds informations about the game state
\item the \textbf{statsKeeper} class keeps track of the opponent models
\item Information class, that holds all the information of the tables but also for each player
\end{itemize}
Preferably each table runs on it's own thread, a coordinator makes sure that tables wait if reseating has to be done.
\subsection{Open Bugs and Fixes}
\begin{itemize}
\item is Raise action for > stack a valid allInAction? -> \textbf{for now it is}
\item allInAction for less than a real raise should be treated like \textbf{call + extra}, meaning, the lastRaiseAmount stays the same but a new Raise has to be twice the lastRaiseAmount + extra
\item if on flop two all in and two active players, and the first active player folds -> second active player shouldn't have the opportunity to doTurn
\item change the way and all-in action is treaded (for now a raise of > stack size is not treated as all-in, it has to be exactly the same)
\end{itemize}

\subsection{Things To Implement}
\begin{itemize}
\item A UnitTest likle in oopoker\_master which tests importent functions for correctness
\item porting the neural network from python to c++ (using tensorflow api or write own neural network?)
\item \textbf{Neural Network:} how to normalize input values if distribution is unknown???\\\\
\textbf{Answer:}\\\\
Additional feature (totalchips in game as BB), and other chip features also in bb
Additionally normalize all chip features with the totalchips feature in bb and normalize other features to 0...1. 
\item Last layer is a softmax activation function so it gets a classification characteristic
\item \textbf{Evolutionary Algorithm:} How to evolve our agents?
\item how to order agents by places when multiple agents bust within one hand?
\subitem \textbf{Answer:} guy with less chips has worst position...
\item \textbf{Effective Hand Strength} as feature -> see how it is calculated and then try to do it with pokereval.h and pokereval2.h
\subitem https://en.wikipedia.org/wiki/Poker\_Effective\_Hand\_Strength\_(EHS)\_algorithm
\subitem \textbf{github for EHS calculation algo: https://github.com/Pip3r4o/Bluffasaurus}
\item interesting read on EHS for multiple opponents and general implementation strategies.. 
http://poker-ai.org/archive/www.pokerai.org/pf3/viewtopicfdcf.html?f=3\&t=444\&st=0\&sk=t\&sd=a\&start=40
\end{itemize}

\pagebreak
%%%%%%%%%%%%%%%%%%%%%%%%%%%%%%%%%%%%%%%%%
\section{Game Play Plan}
\textbf{Pregame}
\begin{itemize}
\item get number of players
\item calculate number of tables and cardDealers
\item give players a buy-in and assign them to tables
\end{itemize}
Once all players have their buy-in and seat on a table, the game can start\\\\
\textbf{Game Start}\\\\
\textit{Preflop}
\begin{itemize}
\item \textbf{Dealer:} assign dealer button to player
\item \textbf{Players:} post SB and BB and antes
\item \textbf{Dealer:} shuffle deck of cards
\item \textbf{Dealer:} deal 2 cards to each player
\item \textbf{Dealer:} tell first player after BB to play action
\subitem \textbf{Players:} make a move 
\subitem \textbf{Dealer:} check if action is valid
\subitem \textbf{Dealer:} repeat with next player 
\item \textbf{Dealer:} after all players acted:
\subitem calculate statistics 
\subitem update table
\end{itemize}

\textit{Flop}
\begin{itemize}
\item \textbf{Dealer:} burn one card, and deal flop
\item \textbf{Dealer:} tell first player (SB) to play action
\subitem \textbf{Players:} make a move 
\subitem \textbf{Dealer:} check if action is valid
\subitem \textbf{Dealer:} repeat with next player 
\item \textbf{Dealer:} after all players acted:
\subitem calculate statistics 
\subitem update table
\end{itemize}
%\subitem send information to game (communicate with game, i.e ask if should wait...)
\pagebreak
\section{Game Assets Classes}
\subsection{Information Class}



% **************************************************************************************************
% **************************************************************************************************

% place all floats and create label on last page
\FloatBarrier\label{end-of-document}
\end{document}

