%%%% Time-stamp: <2018-03-24 14:05:09 vk>
%% ========================================================================
%%%% Disclaimer
%% ========================================================================
%%
%% created by
%%
%%      Karl Voit
%%

%% ========================================================================
%%%% Basic settings
%% ========================================================================
%% (idea of using newcommands for basic documentclass settings from: Thomas Schlager)

\newcommand{\mypapersize}{A4}
%% e.g., "A4", "letter", "legal", "executive", ...
%% The size of the paper of the resulting PDF file.

\newcommand{\mylaterality}{oneside}
%% "oneside" or "twoside"
%% Either you are creating a document which is printed on both, left pages
%% and right pages (twoside) or you create a document which is printed
%% on right pages only (oneside).

\newcommand{\mydraft}{false}
%% "true" or "false"
%% Use draft mode? If true, included graphics are replaced by empty
%% rectangles (of same size) and overfull boxes (in margin space) are
%% marked with black box (-> easy to spot!)

\newcommand{\myparskip}{half}
%% e.g., "no", "full", "half", ...
%% How to separate paragraphs: indention ("no") or spacing ("half",
%% "full", ...).

\newcommand{\myBCOR}{0mm}
%% Inner binding correction. This value depends on the method which is
%% being used to bind your printed result. Some techniques do not
%% require a binding correction at all ("0mm"), other require for
%% example "5mm". Refer to KOMA script documentation for a detailed
%% explanation what a binding correction is and how to measure it.

\newcommand{\myfontsize}{12pt}
%% e.g., 10pt, 11pt, 12pt
%% The font size of the main text in pt (points).

\newcommand{\mylinespread}{1.5}
%% e.g., 1.0, 1.5, 2.0
%% Line spacing in %/100. For example 1.5 means 150% of the usual line
%% spacing. Please use with caution: 100% ("1.0") is fine because the
%% font was designed for it.

\newcommand{\mylanguage}{ngerman,american}
%% "english,ngerman", "ngerman,english", ...
%% NOTE: The *last* language is the active one!
%% See babel documentation for further details.

%% BibLaTeX-settings: (see biblatex reference for further description)
\newcommand{\mybiblatexstyle}{authoryear}
%% e.g., "alphabetic", "authoryear", ...
%% The biblatex style which is being used for referencing. See
%% biblatex documentation for further details and more values.
%%
%% CAUTION: if you change the style, please check for (in)compatible
%%          "biblatex" package options in the file
%%          "template/preamble.tex"! For example: "alphabetic" does
%%          not have an option "dashed=..." and causes an error if it
%%          does not get removed from the list of options.

\newcommand{\mybiblatexdashed}{false}  %% "true" or "false"
%% If true: replace recurring reference authors with a dash.

\newcommand{\mybiblatexbackref}{true}  %% "true" or "false"
%% If true: create backward links from reference to citations.

\newcommand{\mybiblatexfile}{references-biblatex.bib}
%% Name of the biblatex file that holds the references.

\newcommand{\mydispositioncolor}{40,40,40}
%%\newcommand{\mydispositioncolor}{30,103,182}
%% e.g., "30,103,182" (blue/turquois), "0,0,0" (black), ...
%% Color of the headings and so forth in RGB (red,green,blue) values.
%% NOTE: if you are using "0,0,0" for black, printers might still
%%       recognize pages as color pages. In case this is a problem
%%       (paying for color print-outs vs. paying for b/w-printouts)
%%       please edit file "template/preamble.tex" and change
%%       "\definecolor{DispositionColor}{RGB}{\mydispositioncolor}"
%%       to "\definecolor{DispositionColor}{gray}{0}" and thus
%%       overwriting the value of \mydispositioncolor above.

\newcommand{\mycolorlinks}{true}  %% "true" or "false"
%% Enables or disables colored links (hyperref package).

%\newcommand{\mytitlepage}{template/title_Thesis_TU_Graz}
\newcommand{\mytitlepage}{template/title_Thesis_TU_Graz}
%% Your own or one of following pre-defined title pages:
%% "template/title_plain_maketitle": simple maketitle page
%% "template/title_Diplomarbeit_KF_Uni_Graz.tex": fancy (german) title page for KF Uni Graz
%% "template/title_Thesis_TU_Graz":
%%             titlepage for Graz University of Technology (correct
%%             (old?) Corporate Design) by Karl Voit (2012)
%% "template/title_Thesis_TU_Graz_-_kazemakase":
%%             titlepage for Graz University of Technology
%%             (correct new Corporate Design) by kazemakase (2013):
%%             see https://github.com/novoid/LaTeX-KOMA-template/issues/5
%% "template/title_VWA": titlepage for Vorwissenschaftliche Arbeit

\newcommand{\mytodonotesoptions}{}
%% e.g., "" (empty), "disable", ...
%% Options for the todonotes-package. If "disable", all todonotes will
%% be hidden (including listoftodos).

%% Load main settings for document preamble:
\input{template/preamble}%% DO NOT REMOVE THIS LINE!

\setboolean{myaddcolophon}{false}  %% "true" or "false"
%% If set to "true": a colophon (with notes about this document
%% template, LaTeX, ...) is added after the title page.
%% Please do not set to "false" without a good reason. The colophon
%% helps your readers to get in touch with LaTeX and to find this template.

\setboolean{english_affidavit}{true}  %% "true" or "false"
%% If set to "true": the language of the statutory declaration text is set to
%% English, otherwise it is in German.


%% ========================================================================
%%%% Document metadata
%% ========================================================================

%% general metadata:
\newcommand{\myauthor}{FABIAN MOIK}  %% also used for PDF metadata (hyperref)
\newcommand{\myauthorwithexistingtitles}{\myauthor{}}  %% including
                                %% university degree already held
                                %% (BSc, MSc, ...)
\newcommand{\mytitle}{TITLE}  %% also used for PDF metadata (hyperref)
\newcommand{\mysubtitle}{ }  %% only used with title_Thesis_TU_Graz_-_kazemakase
\newcommand{\mysubject}{SUBJECT}  %% also used for PDF metadata (hyperref)
\newcommand{\mykeywords}{KEYWORDS}  %% also used for PDF metadata (hyperref)

%% this information is used only for generating the title page:
\newcommand{\myworktitle}{Bachelor's Thesis}  %% official type of work like ``Master theses''
\newcommand{\mygrade}{Bachelor of Science} %% title you are getting with this work like ``Master of ...''
\newcommand{\mystudy}{Information and Computer Engineering} %% your study like ``Arts''
\newcommand{\mydegreeprogramme}{Bachelors's degree programme: \mystudy} %% Master's or PhD degree programme
\newcommand{\myuniversity}{Graz University of Technology} %% your university/school
\newcommand{\myfaculty}{ }  %% only used with title_Thesis_TU_Graz_-_kazemakase
\newcommand{\myinstitute}{Institute for Signal Processing and Speech Communication} %% affiliation
\newcommand{\myinstitutehead}{Univ.-Prof.\,Dipl-Ing.\,Dr.techn.~Some One} %% head of institute
\newcommand{\mysupervisor}{Assoc.Prof. Dipl.-Ing. Dr.mont. Pernkopf Franz} %% your supervisor
\newcommand{\mycosupervisor}{\ }  %% only used with title_Thesis_TU_Graz_-_kazemakase
\newcommand{\myevaluator}{Prof.~Some Genius} %% your evaluator
\newcommand{\myhomestreet}{Brucknerstra{\ss}e 11} %% your home street (with house number)
\newcommand{\myhometown}{Graz} %% your home town
\newcommand{\myhomepostalnumber}{8010} %% your postal number of home town
\newcommand{\mysubmissionmonth}{Oktober} %% month you are handing in
\newcommand{\mysubmissionyear}{2018} %% year you are handing in
\newcommand{\mysubmissiontown}{\myhometown} %% town of handing in (or \myhometown)


%% additional information for generic_documentation title page
\newcommand{\myid}{01430095} %% Matrikelnummer
\newcommand{\mylecture}{LECTURE} %%

%%%% TODOs

\newcommand{\TODOinsertref}[1]{\todo[color=gray!40]{#1}}
\newcommand{\TODOexplainindetail}[1]{\todo[color=green!40]{#1}}
\newcommand{\TODOrewrite}[1]{\todo[color=red!40]{#1}}

\newcommand{\markred}{\textcolor{red}}
\newcommand{\markblue}{\textcolor{blue}}
\newcommand{\pokerterm}[1]{\textbf{\textit{#1}}}

%% ========================================================================
%%%% MISC command definitions
%% ========================================================================
\input{template/mycommands}

%% ========================================================================
%%%% Typographic settings
%% ========================================================================
\input{template/typographic_settings}


%% ========================================================================
%%%% MISC usepackages
%% ========================================================================

%% ... it's OK to put here your own usepackage commands ...




%% ========================================================================
%%%% MISC self-defined commands and settings
%% ========================================================================

%% ... it's OK to put here your own newcommand/newenvironment-definitions ...




\newcommand{\myLaT}{\LaTeX{}@TUG\xspace} %% LaTeX@TUG text "logo"

\hyphenation{ex-am-ple hy-phen-ate}  %% in order to use German umlauts
%% here (Ver-\"of-fent-li-chung), you have to check for
%% activated \usepackage[T1]{fontenc} in the preamble

%% override default language of babel: (be sure to know, what you're
%% doing here)
%\selectlanguage{american}
%\selectlanguage{ngerman}

%% ========================================================================
%%%% Templates
%% ========================================================================

%% template for inserting figures:
% \myfig{}%% filename
%       {}%% width/height
%       {}%% caption
%       {}%% optional (short) caption for list of figures
%       {fig:}%% label

%% acronyms in small caps: \myacro{UNESCO}


\input{template/pdf_settings}  %% should be *last* definitions in preamble!
%% ========================================================================
%%%% begin{document}
%% ========================================================================
\begin{document}

\frontmatter                    %% KOMA: roman page numbers and such; only available in scrbook

\input{colophon}                %% defines information about editor, LaTeX, font, ...

%% Choose your desired title page:
\input{\mytitlepage}            %% include title page


\input{template/declaration_TU_Graz}  %% Statutory Declaration
% \input{thanks}                %% this is a suggestion: you have to create this file on demand
% \input{foreword}              %% this is a suggestion: you have to create this file on demand


%% include the abstract without chapter number but include it on table of contents:
\cleardoublepage
\phantomsection
\addcontentsline{toc}{chapter}{Abstract}
%%%% Time-stamp: <2013-02-25 10:31:01 vk>


\chapter*{Abstract}
\label{cha:abstract}

Poker is a card game with imperfect information where players have to deal with randomness, hidden information, opponent modeling, risk management and deception. These properties turn the game into a very interesting test-bed for artificial intelligence research. Evolutionary algorithms have been used in many fields of machine learning to find solutions to problems in a large decision space, while neural networks nowadays are widely used to find solutions in non-linear decision spaces \cite{challenge_of_poker, evolutionary_methods}.\par
A combined version of these two models is used in this thesis to create No-Limit Texas Hold'em poker agents capable of developing a profitable playing style and learning the fundamental principles of a successful poker strategy. To counter some problems inherent in evolutionary algorithms, such as \textit{Evolutionary Forgetting} a concept called \textit{Hall of Fame} is used to improve the performance of the evolved agents. 
\textit{Opponent modeling} is an essential part of the decision-making process of a poker player and is imperative to achieve a high skill in the game. The results of the conducted experiments show that a hall of fame greatly increases the overall performance of agents, while opponent modeling allowed to develop a versatile strategy that works especially well against static opponent types. The resulting program successfully understood the most important principles for playing profitable poker but there remains further research to be done to achieve the skill of professional tournament poker players.
%\glsresetall %% all glossary entries should be used in long form (again)
%% vim:foldmethod=expr
%% vim:fde=getline(v\:lnum)=~'^%%%%\ .\\+'?'>1'\:'='
%%% Local Variables:
%%% mode: latex
%%% mode: auto-fill
%%% mode: flyspell
%%% eval: (ispell-change-dictionary "en_US")
%%% TeX-master: "main"
%%% End:
              %% Abstract


\tableofcontents                %% this produces the table of contents - you might have guessed :-)

\listoffigures
\listoftables

\mainmatter                     %% KOMA: marks main part using arabic page numbers and such; only available in scrbook


%\input{example-short-chapter}   %% remove this line to get rid of the example chapter
%\input{example-style-chapter}   %% remove this line to get rid of the style chapter

%% include tex file chapters:
%%%% Time-stamp: <2013-02-25 10:31:01 vk>


\chapter{Introduction}
\label{cha:introduction}

Until recently artificial intelligence (AI) research has focused on creating intelligent algorithms capable of winning against the best human players in a variety of board games such as \textit{chess}, \textit{backgammon} or \textit{go}. In these board games players have perfect information about the entire game state at any given moment, which allows algorithms to brute-force calculate an optimal strategy. Card games, such as \textit{poker} however do not provide perfect information about the game state and therefore provide a much more challenging environment for intelligent algorithms. The game of poker has several interesting characteristics including \textit{imperfect information}, \textit{risk management}, \textit{opponent modeling} and \textit{deception}, that make it an ideal candidate for AI research. Developing a world class artificial poker agent may also help solving open questions in AI research and provide important research benefits. \cite{challenge_of_poker, pena}.\par
 The goal of this thesis is it to develop an artificial poker agent capable of understanding the fundament concepts of \textit{No-Limit Texas Hold'em poker}, to establish a profitable strategy against \textit{static opponents}, by applying \textit{evolutionary algorithms} to an artificial neural network.\par
 Chapter 2 of this thesis gives an introduction to the game of poker, focusing on the \textit{No-Limit} variant of \textit{Texas Hold'em}, and explains the basic rules of the game. Chapter 3 provides an overview of previous work done in the field of computer poker, especially focusing on the creation of artificial poker bots. Chapter 4 discusses the design and implementation of the evolutionary neural network poker bot and explains some key components to achieve a high level poker skill. Moreover the procedure of training and evaluating the neural network agents is explained in detail in this chapter. In Chapter 5 experimental results and interesting observations are discussed. Chapter 6 summarizes conclusions taken from this work and provides possible future work in this field. 

%\glsresetall %% all glossary entries should be used in long form (again)
%% vim:foldmethod=expr
%% vim:fde=getline(v\:lnum)=~'^%%%%\ .\\+'?'>1'\:'='
%%% Local Variables:
%%% mode: latex
%%% mode: auto-fill
%%% mode: flyspell
%%% eval: (ispell-change-dictionary "en_US")
%%% TeX-master: "main"
%%% End:
     
%%%% Time-stamp: <2013-02-25 10:31:01 vk>


\chapter{Poker Basics}
\label{cha:pokerBasics}
\section{What is Poker}
In David Sklansky's \cite{theory_of_poker} poker is being described as a term for a whole family of card games, which can be grouped into different types. In some game variants such as  \textit{Texas Hold'em or Omaha} the player with the highest \pokerterm{hand rank}\TODOrewrite{Rewrite the footnote}\footnote{All poker specific terms mentioned the first time, appear in bold, italics throughout this bachelor's thesis. They are defined and briefly explained in the Appendix.} wins, in others such as \textit{Razz or Badugi} the player with the lowest hand rank wins, and in yet others the \pokerterm{pot} is split between the player with the highest and the player with the lowest hand rank \TODOrewrite{Check if citation is correct}.\par
Among all poker variants there are different \pokerterm{betting} structures. Many poker formats offer \pokerterm{limit} variants of the game and \pokerterm{no-limit} variants. In limit games the \pokerterm{bet size} has an upper and lower limit, whereas in no-limit games the bet size is just restricted by the amount a player has currently left in his \pokerterm{stack} \cite{theory_of_poker}. While bet sizes may not be restricted in no-limit games there are specific rules (\markred{throughout}) all poker variants that determine the minimum bet size at any given time in the game regardless of the variant of the game itself.
\section{No-Limit (NL) Texas Hold'em}
The focus of this work is drawn to the \pokerterm{no-limit} version of the Texas Hold'em poker variant. Not only is it the most widely played poker variant but also considered to be the most challenging form of poker \cite{strong_poker}. While finding the optimal strategy for Texas Hold'em poker may be complex it has exceptionally simple rules \cite{billings_phd}.
\subsection{Poker Rules}
A Texas Hold'em \pokerterm{hand} begins in a stage called \pokerterm{pre-flop}. In this stage every player at the table is dealt two cards. The cards are dealt face down and are exclusive to the player recieving the cards \cite{billings_phd}. Each seat at the table has a special meaning throughout a poker hand. The seat (also called \pokerterm{position}) that (\markred{that or which}) owns the \pokerterm{dealer-button} determines the two positions that have to place  \pokerterm{forced bets} called the \pokerterm{small blind} and the \pokerterm{big blind}. This dealer-button moves clockwise by one position once a whole poker hand is finished. The player directly to the left of the \pokerterm{dealer}, the player currently holding the dealer-button, is forced to bet one small blind and the player two seats to the left of the dealer has to bet one big blind, the equivalent of two small blinds \cite{master_nuno}. These (\markred{these or those?}) bets serve the purpose of preventing players to wait for the best hand without any punishment. \par In case of a Texas Hold'em poker tournament most often another type of forced bet called the \pokerterm{ante} is present in every hand of the tournament \cite{poker_dummies}. It usually is a percentage portion of the current big blind in the range of 10\%-20\% that has to be contributed to the \pokerterm{pot} by each player.
In poker tournaments the small blind and the big blind are incremented after a fixed time interval to advance the game.  \par
The pre-flop stage is concluded once each player has acted in the first betting round. The next stage, called the \pokerterm{flop}, is introduced by three face up dealt cards to the \pokerterm{community board} and players once again have the chance to bet in this round. The third stage is called the \pokerterm{turn} and a fourth card is dealt to the community board. After concluding the third round of betting a fifth \pokerterm{community card} is dealt on the \pokerterm{river} and players are allowed to bet one last time. If there is more then one player left after this final round of betting the cards of those (\markred{those or these}) players are revealed in a so-called \pokerterm{showdown}. The best combination of two player's \pokerterm{hole cards} and three community cards decides the winner. If two players have the same hand rank, the pot is split between both of them \cite{billings_phd}.
\subsection{Betting}
Sophisticated betting strategies are the number one key component for succeeding in the game of NL Texas Hold'em poker. Choosing the correct (\markred{There is no -correct- size, rewrite this}) bet size in any given situation allows players to maximize their winnings and at the same time minimize their losses \cite{master_nuno}. \par
Each betting round is either opened by a bet or a check. In a clockwise manner players may then decide to check or bet if there was no bet placed yet. If a bet has already been placed, the remaining players may either call the bet, raise the bet or fold to that bet. In NL Texas Hold'em the bet or raise amount is only limited by the remaining chips a player has \cite{poker_dummies}. \par
\subsubsection{Betting Options}
\begin{itemize}
\item \textbf{Check / Call}:
A check describes the action of staying in the hand without committing any money to the pot. To check a hand there must be no previous bets in the current betting round. By calling someone's bet a player wagers the amount of chips to equalize the amount of chips committed to the pot by each player.\\ \TODOrewrite{This needs to be more clear}
\item \textbf{Bet / Raise}:
A bet describes the action of a player being first to put a certain amount of money into the pot. If there already exist a bet a player may still put more money into the pot then the previous bets did. This is referred to as a raise. \\ \TODOrewrite{This needs to be more clear}
\item \textbf{Fold}:
Folding a hand means no longer being interest in the hand and not willing to put any more money into the pot. If however no bet has been placed in the current betting round yet, a player may rather check his hand instead of folding it. \TODOrewrite{This needs to be more clear}
\end{itemize}
\cite{review, poker_dummies} 
\subsection{\markred{Hand Rankings}}
Texas Hold'em poker is a poker variant where the highest ranking 5-card combination wins \cite{poker_dummies}. To decide the winner of a poker hand the best possible 5-card combination of each remaining player is compared against each other \cite{pena}. The ranking system of 5-card combinations can be found in \markred{Table 2.1...}. 
\TODOrewrite{Breifly explain the main concept of ranking hands and then link to the table.} 
\section{What Defines a GOOD Poker Player}
\markred{Maybe transition this section to the IMPLEMENTATION, to explain how it was implemented then.}
%\glsresetall %% all glossary entries should be used in long form (again)
%% vim:foldmethod=expr
%% vim:fde=getline(v\:lnum)=~'^%%%%\ .\\+'?'>1'\:'='
%%% Local Variables:
%%% mode: latex
%%% mode: auto-fill
%%% mode: flyspell
%%% eval: (ispell-change-dictionary "en_US")
%%% TeX-master: "main"
%%% End:
        
%%%% Time-stamp: <2013-02-25 10:31:01 vk>


\chapter{State-Of-The-Art}
\label{cha:State-Of-The-Art}
The game of poker has many interesting properties which proved to offer a challenging test environment for artificial intelligence research. Over the last decade scientists and researchers have studied the game from a game theoretical view to find optimal solutions but also tried a number of machine learning algorithms and artificial intelligence systems on the game \cite{opp_master, challenge_of_poker}. A lot of research though focused on simplified versions of the game because these variants of poker are easier to analyze but still offer demonstrations of game theoretical principles \cite{challenge_of_poker}. Nonetheless over the past few years, abstract versions of poker have been used to first successfully train algorithms on this simple version of the game and then transfer the obtained knowledge to a full version of the game \cite{opp_master, challenge_of_poker}.
\section{Knowledge-based Poker Agents}
Knowledge-based systems can be broken into two categories, namely \textbf{rule-based expert systems} and \textbf{formula-based methods}. In general knowledge-based systems require the knowledge of an expert player to design the system \cite{review}. A \textit{rule-based expert system} in its simplest form is a sequence of if-else statements for frequently occurring scenarios of the game which determine the desirable \pokerterm{betting action}. More advanced human players describe their poker hands in a very similar way when they break them down in a discussion \cite{master_nuno}. \textit{Formula-based methods} on the other hand try to generalize the problem by defining a formula that takes a set of inputs and outputs a so-called \pokerterm{probability triple} upon which a betting decision is made. Inputs to the formula may describe important information about the current game state and can be weighted to reinforce the importance of certain inputs over others \cite{review}. \par
While rule-based expert systems may yield reasonably good results in the first stage of a poker hand (pre-flop), they fail to shine in later stages of the game because they become increasingly difficult to maintain with additional and sometimes even conflicting information added at each stage. Furthermore  static strategies are prone to exploitation and therefore not competitive with other approaches \cite{review}.
\section{Simulation-based Poker Agents}
In general simulation based methods rely on repeatedly simulating an outcome in order to approximate the resulting expected value of an action \cite{master_nuno}. A frequently used simulation method applied to the game of poker is the so called \textit{Monte-Carlo simulation} or \textit{Monte-Carlo Tree Search}, which is a procedure that searches the game tree by sampling the possible choices in a game state and simulates the action taken to the bottom of the tree. By repeating this procedure a robust expectation value can be computed \cite{review}. 
In Billings \textit{et. al} \cite{selective_sampling}, the technique of \textit{selective sampling} is described, which uses information about the opponents to bias the selection of possible card combinations a player may hold. With this modification to the sampling algorithm they were able to create a more dynamic betting strategy due to the gain of valuable information about opponents. 
\section{Game-Theoretic Optimal Poker Agents}
The currently best performing poker-playing programs are approximating a Nash equilibrium \cite{quality_of_bots}.
In game theory a Nash equilibrium describes a state of a game where no player can find an action that would yield a better outcome than the suggested equilibrium action, given that all other players also choose to take the suggested action \cite{game_theory}. \par
This strategy has proven to achieve great results in zero-sum games, like NL \pokerterm{heads-up} poker \cite{master_nuno}. Most successful heads-up poker bots use an abstract version of the poker variant in which they approximate the Nash equilibrium and then transfer the decision made in the abstract version to the real version of the game. One of the most successful algorithm in approximating the Nash equilibrium in an abstract version of the game is called \textit{Counterfactual Regret Minimization}. The top three poker bots in the \textit{Annual Computer Poker Competition} (ACPC) 2016 used a variant of the Counterfactual Regret (CFR) algorithm to defeat their competitors \cite{quality_of_bots}.
\section{Adaptive / Exploitive Poker Agents}
Nash equilibrium approaches and other static poker strategies are vulnerable to exploitation \cite{master_nuno}.
Adaptive strategies try to tackle this problem by quickly adapting to the playing-style of the opponents and exploiting their weaknesses. Two algorithms, namely the \textit{Miximax} and \textit{Miximix} algorithm achieve this by searching an adaptive imperfect information game tree. Decisions are then made by considering a \textit{randomized mixed strategy} associated with the decision node of the searched tree \cite{billings_phd}.%\TODOexplainindetail{Some more information on the success of those strategies?}
\section{Bayesian Poker Agents and Evolutionary Algorithms}
\subsection{Bayesian Poker Agents}
A Bayesian network is a probabilistic graphical model. Each node in the directed acyclic graph represents a random variable and edges between nodes represent the conditional dependencies of variables. Nodes are associated with a probability function which returns the conditional probability values based on their parent's values. A probability distribution over the random variables can be retrieved by propagating the probabilities of initialized nodes throughout the network \cite{review}. \par
Compared to other poker playing agents, bayesian agents performed badly in the AAAI Computer Poker Competitions in the last years. There is still a lot of room for improvement in Bayesian based networks and further research in this field has to be done to be competitive in upcoming competitions \cite{review}. 
\subsection{Evolutionary Algorithms}
Evolutionary algorithms try to mimic the evolution process of biological bodies. Agents of one generation are competing in a population to achieve a good score for a given task. The best agents are then evaluated and chosen as parents for the next generation. Some small changes are applied to the genetic information of the newly created offsprings before they compete with the best agents of the previous generation in a new population. This process repeats for a number of generations and optimally should yield a global increase of performance \cite{evolutionary_methods}.\par
Over the last years evolutionary algorithms were used in some studies to train artificial neural networks to play poker. Nicolai and Hilderman \cite{nn_evolve} train neural network agents to learn the game of poker and propose some methods to counter problems inherent in evolutionary algorithms. While poker agents trained with an evolutionary algorithm have not yet performed very well against other poker bots, there is still a lot to learn and discover from evolutionary algorithms applied to neural networks in the future \cite{review}. 
%\glsresetall %% all glossary entries should be used in long form (again)
%% vim:foldmethod=expr
%% vim:fde=getline(v\:lnum)=~'^%%%%\ .\\+'?'>1'\:'='
%%% Local Variables:
%%% mode: latex
%%% mode: auto-fill
%%% mode: flyspell
%%% eval: (ispell-change-dictionary "en_US")
%%% TeX-master: "main"
%%% End:
        
%%%% Time-stamp: <2013-02-25 10:31:01 vk>


\chapter{Implementation - Structure of the Bot}
\label{cha:implementation}
%%% Introduction
In this chapter the architecture of both the test environment and the poker playing agents is described and the learning process of the poker agents is explained. As mentioned in \markred{chapter X (ref to it)} there are some key components that distinguish a strong poker player from a weak one. A well defined \textit{betting strategy} both attuned to the mathematical principles such as \textit{hand strength} and \textit{hand potential} and to the interpretation of opponent's tendencies decides on the profitability of a poker player over the \pokerterm{long run} \cite{opp_modeling}. This chapter describes a possible way to implement and apply these concepts to create a poker playing agent, capable of understanding the situation on the table and assessing it's hand strength versus it's opponents. \TODOrewrite{should I use it's her for agent? or better his/her?}

%%%%%%%%%%%%%%%%%% ARCHITECTURE %%%%%%%%%%%%%%%%%%
\section{Architecture \markred{of Testbed}}
\begin{itemize}
\item Why C++ and not an existing testbed?
\subitem name reasons why
\item briefly explain how the testbed is build up (Main classes and function of an object $\rightarrow$ Dealer -> deals cards and handles gamestate information, reduces players from pool, etc... \markred{have a look into code structure and make diagram}.
\end{itemize}
%%%%%%%%%%%%%%%%%% NN AGENTS %%%%%%%%%%%%%%%%%%%%%%
\section{Neural Network Agents}
%%%%%%%%%%%%%%%%%% BETTING STRATEGY %%%%%%%%%%%%%%%%%%%%%%
\section{Betting Strategy}
\begin{itemize}
\item introduction sentence
\item say that our system was influenced by paper \cite{opp_modeling}
\item tell that there are 169 distinct hand types, to find strength of the hands preflop multiple possibilities (Sklansky's group, rollout simulation, or our method) \markred{this is subsection preflop strategy}
\item postflop strategy:  to assess the strength of a hand, a hand evaluation is needed
\subitem consists of hand ranker, hand strength and potential estimations, forming EHS
\subitem hand ranker
\subitem hand strength
\subitem hand potential 
\subitem effective handstrength
\subsubitem EHS for multiple opponents
\item betting strategy upon all those factors -> we did probability curves, depending on output of NN agent? To become a little bit more unpredictable 
\end{itemize}
%%%% %%%%%%%%% PREFLOP
\subsection{Preflop Strategy}
%%%%%%% HAND RANKER
\subsubsection{Hand Ranker}

\begin{itemize}
\item created pre computed table with all possible hand types 
\subitem say how many hands there are but that they can be grouped into 169 distinct hand types (give example)
\subitem 1000000 off-line simulations against 1-8 opponents.
\subitem explain which evaluator was used and why
\end{itemize}
Due to the fact, that there is a limited number of possible preflop states a precalculated table was computed to obtain an approximation on the win percentage of a specific hand against a specific number of opponents. This approach does not take opponent model into consideration an. \par
For evaluating the effective hand strength of one agent against $N$ opponents a lookup table was created. It consists of 1352 entries. The hand of the agent is index by it's strength, meaning that an index of 0 is the best possible holding ($AA$) and an index of 168 is the weakest possible holding ($72o$). The indexing tough was arbitrarily choosen to be as stated. In total there are 169 * 8 possible preflop situations a player could find himself in. For all those situation a \textit{Monte-Carlo Simulation} (\markred{performed on the TPTEvaluator, which evaluated the simulated 7 card situations}) was performed to get an approximation of the expected win percentage of each possible hand against each possible number of opponents. The values are saved in a lookup table which allows the fastest possible access time without any further need of calculations. \par
The win percentage represent the chance of a hand beating another hand under the assumption of seeing all 5 community cards and going to showdown.\TODOexplainindetail{Does this represent the HS or the EHS?}
%%%% POSTFLOP
\subsection{Postflop Strategy}
\begin{itemize}
\item introduction sentence that hand evaluation is needed and why it is important to have a fast one, say that 
\item subsubsection(Hand evaluation): explain which evaluator/ranker was used and why
\item explain purpose of it
\item list Hand Strength and Hand Potential as subsections as in \cite{opp_master}
\item EHS
\end{itemize}
\markred{Guess we did not use hand potential but rather just pure hand strength??? Why? We already have the implementation of EHS in code but just don't use it! why? Maybe it is faster to use 2000 samples on the full runout and it approximates the EHS well enough? $\rightarrow$ TEST THIS}
For evaluating the strength of a hand after the preflop stage, it is important to not only consider the immediate hand strength but also the positive and negative hand potential. This metrics combined yield the so-called \textbf{Effective Hand Strength (EHS)}. To calculate the EHS of an agent at a given stage after the preflop stage (flop, turn, river), a Monte-Carlo simulation was run 2000 times, simulating possible runouts
%%%%%%%%%%%%% NN %%%%%%%%%%%%%
\section{Training the NN-Agents}
\pagebreak
\begin{itemize}
\item introduction
\subitem outlook to what follows now
\subitem mention that limitations were found and that the chapter describes how the limitations are or can be tackled.
\item Architecture
\subitem language used and why
\subitem short testbed description and why others where not used
\subitem make a graphic which shows how the system is structured and explain the system components
\subsubitem brief explanation on what the Hand Ranker/Hand Evaluator does $\rightarrow$ reference to the upcoming in depth explaination
\subsubitem NN agent (brief explanation of basic rule $\rightarrow$ in depth explanation follows)
\item Neural Network Agent
\subitem general structure
\subitem Feature List explained
\item Betting Strategy
\subitem Hand ranker and Hand Evaluator (Effective Hand Strength and Hand Potential)
\subitem preflop and postflop decision making
\subsubitem Monte Carlo Simulation
\subitem betting decision made according to the probability triple returned by the NN agents.
\subsubitem explain how the action is choosen (bet size curves etc.)
\item Training the Poker Agent
\subitem Evolutionary algorithm
\subitem Hall of Fame
\subitem Co-evolution?
\end{itemize}
%\glsresetall %% all glossary entries should be used in long form (again)
%% vim:foldmethod=expr
%% vim:fde=getline(v\:lnum)=~'^%%%%\ .\\+'?'>1'\:'='
%%% Local Variables:
%%% mode: latex
%%% mode: auto-fill
%%% mode: flyspell
%%% eval: (ispell-change-dictionary "en_US")
%%% TeX-master: "main"
%%% End:
        
%%%% Time-stamp: <2013-02-25 10:31:01 vk>


\chapter{Experimental Results}
\label{cha:results}

%%%% INTRODUCTION
In this chapter, the results of three different methods for training an evolutionary neural network poker agent are presented. In the first experiment a baseline control agent was created using an evolutionary neural network without any countermeasures accounting for problems with evolutionary algorithms. In the second and third experiment a hall of fame was introduced to keep strategies, which have proven to be strong in previous generations, in the playing population. Additionally, in the third experiment playing tendencies of opponents were modeled and given as input to the neural network, with the goal to further improve the ability to adjust to certain playing styles. \par
The skill of the best agent after a number of generations was then evaluated with two different metrics against a set of static opponents for each experiment. \markred{Something else to mention here?}
%%TODO write like in ENN_Read.pdf
\section{Benchmark Opponents}
% Shortly explain here that we benchmarked against static opponents.
To test the skill of evolved neural network agents, they played in a number of tournaments against predefined static opponents. In a series of tournaments all players are ranked by the fitness function described in Subsection \markred{x.x}. For all benchmark tests the same weight distribution for the fitness function was used to make the results of different evolution methods comparable to each other. While the \textit{average placement} in a tournament might be a strong indicator for the level of skill of a poker player, it certainly should not be used as a benchmark value on its own. In this thesis a combination of three benchmark values was used to assess the skill of an agent. Furthermore another fitness function is later used to determine the profitability of a poker agent by calculating the \textit{return of investment} over a series of tournaments. This fitness function is often used in real live poker because success of tournament players is measured by the amount of money they won in their poker career.
% Talk about the weights for each single value and that they were found to be well suiting and were used throughout the testing of all agents

%In a static environment where all players are following a static rule set and do not exploit the weaknesses of their opponents, a \textit{folding strategy} is really strong.
\subsection{Always Fold}
An \textit{Always Fold} agent does exactly what his name suggests, he always folds his two hole cards when it is his turn to bet. The only exception to this rule is when the action allows to check instead of folding. A folding strategy is very effective in a static poker environment, where agents follow static rules and do not exploit weaknesses of their opponents. While a folding strategy can never win against a betting strategy in tournaments, it might frequently reach high ranks in tournaments. This is because a folding strategy can never lose more chips than the \textit{big blind} in a single hand, while more aggressive strategies frequently bust out of a tournament earlier due to the active betting against opponents.
\subsection{Always Call}
An \textit{Always Call} strategy calls any bet made at the table at any time. If however there is the chance to check, it will do so. 
\subsection{Always Raise}
\textit{Always Raise} agents on the other hand raise a previous bet whenever there is the chance to due so or bet themselves if no previous bet was made yet. 
\subsection{Random}
The \textit{Random} agents implemented in this thesis have a $25\%$ chance of folding or check a hands, $30\%$ chance of raising with a hand, $44\%$ chance of calling with a hand and a $1\%$ chance of going all-in with a hand. The percentage values for each action were arbitrarily chosen to represent a pseudo random behavior.

\section{Evolution without HOF}
In the first conducted experiment 45 randomly created neural network agents played for 200 tournaments per generation. A network topology of \textbf{$16-12-3$} was used for all participating agents. After each generation agents were ranked according to Formula \ref{eq:overall_fitness}, with a weight distribution of $w_1 = 0.8, w_2 = 0.02, w_3 = 0.18$. As indicated by the weights the main focus for agents was to achieve a low \textit{average ranking}, while at the same time maximize the \textit{mean money} won in tournaments. The weight for the \textit{hands won} component of the overall fitness function was considered as being not so important for overall success in tournaments. The \markred{stated} weights for training the neural network agents were established by trial and error and the most fitting set was chosen for all experiments.\par
The best performing 10\% of agents were then selected as possible parents for the evolution phase. This equated to 5 possible parents for 40 newly created offsprings. This new population consisting of the 5 best agents of the previous generation and the 40 new agents then played for another 200 tournaments. This process was repeated for 1000 generations after which the best performing agent was selected to represent a \textit{baseline Control agent}. \par
\subsection{Skill Progression}
Figure \ref{fig:overallfitness_withoutHOF} shows a \textit{moving average} of the progression of skill in the playing population over all 1000 generations. The vertical axis corresponds to the overall fitness of the best performing agent in a given generation. The subset size for the moving average was set to 50 data samples, which yields a smooth representation of the fitness progression.\par
A more detailed view of the individual components of the overall fitness function can be seen in Figure \ref{fig:progression_withoutHOF}, where the fitness is split up into all three components. Each factor is represented as a moving average of size 50 over all generations.\par
\markred{somewhere say that the results were created by simulating 1000 tournaments.}
The \textit{Control} agent was then evaluated versus 44 evenly distributed static opponents. To establish a baseline ranking for all player types, the population consisting of 11 agents for each static opponent type (\textit{Always Fold, Always Call, Always Raise, Random}) and one \textit{Control} agent competed in 100,000 tournaments and were again ranked by their overall fitness as described in Subsection \ref{subsec:fitness}. The results of this baseline evaluation are shown in the histogram in Figure \ref{fig:histo_withoutHOF}. \par
\myfig{Results/withoutHOF/overallfitness.pdf}%% filename in figures folder
  {width=1\textwidth,height=1\textheight}%% maximum width/height, aspect ratio will be kept
  {Overall fitness of \textit{Control} agent over 1000 generations.}%% caption
  {Overall fitness of \textit{Control} agent over 1000 generations.}%% optional (short) caption for table of figures
  {fig:overallfitness_withoutHOF}%% label
\myfig{Results/withoutHOF/progression.pdf}%% filename in figures folder
  {width=1\textwidth,height=1\textheight}%% maximum width/height, aspect ratio will be kept
  {Evolutionary progress for all three individual fitness components over 1000 generations}%% caption
  {Evolutionary progress for all three individual fitness components over 1000 generations}%% optional (short) caption for table of figures
  {fig:progression_withoutHOF}%% label
\myfig{Results/withoutHOF/histogram.pdf}%% filename in figures folder
  {width=1\textwidth,height=1\textheight}%% maximum width/height, aspect ratio will be kept
  {Baseline fitness of all playing styles including the \textit{Control} agent.}%% caption
  {Baseline fitness of all playing styles including the \textit{Control} agent.}%% optional (short) caption for table of figures
  {fig:histo_withoutHOF}%% label
%%%TODO now discuss observed results. The histogram should be discussed and maybe two sentences about the skill progression in the split fitness figure.
Figure \ref{fig:overallfitness_withoutHOF} clearly indicates an overall skill progression over generations. The non-steady increase in skill can be explained by the non-transitive nature of poker, as described in Subsection \ref{subsec:hof}, and by an inherent problem of evolutionary algorithms known as \textit{Evolutionary Forgetting}. In short this phenomenon can be defined as \enquote{the tendency for a population to lose good strategies as they are replaced by seemingly better ones.} \cite[p.63]{evolutionary_methods}. This means that the overall skill level of a generation might decrease due to loosing some strategies, which have proven to be strong. Nonetheless the baseline \textit{Control} agent did beat his competition over 100,000 tournaments, as shown in Figure \ref{fig:histo_withoutHOF}.\par
The best overall fitness of 0.6578 was achieved by the Control agent, closely followed by the static \textit{Always Fold} opponents.\markred{It should be mentioned that the results of the static opponents were averaged}. The \textit{Always Raise} agents performed poorest, with a fitness score of only 0.2467. \textit{Always Call} and \textit{Random} agents also did not perform well, receiving a fitness score of only 0.3124 and 0.3595, respectively. One could think, based on this results, that a folding strategy might be a good choice, which indeed holds true in general, however an \textit{Always Fold} agent will never win a tournament but at best reach second place \cite{evolutionary_methods}.
\subsubsection{Alternative Fitness Evaluation}
 In addition to the \textit{weighted overall fitness} function, which was also used for training the agents, another independent fitness function was evaluated to compare the overall profitability of playing styles. In Figure \ref{fig:roi_withoutHOF} the vertical axis corresponds to the percentage \textit{ROI} for each of the playing styles listed on the horizontal axis.\par
 To calculate the ROI of all agents, the tournament buy-in was set to $1\$$ and a progressive \markred{\pokerterm{payout structure}} shown in Table \ref{tab:payout_structure} was used to reward the seven best ranked players. While \textit{Always Fold} agents achieved an almost equal \textit{overall fitness}, the \textit{Control} agent won far more money over 100,000 tournaments than the second best competitor. With a ROI of $80.27\%$ the \textit{Control} agent's strategy looks far superior compared to a ROI of only $20.52\%$ for the \textit{Always Fold} strategy. However a ROI greater than zero indicates a profitable strategy, which means that only the \textit{Control}, the \textit{Always Fold} and \textit{Always Call} agents were profitable players in the conducted experiment. With a negative ROI of $-73.31\%$ the \textit{Random} strategy performed worst, followed by an almost break even ROI of $-0.9\%$ for the \textit{Always Raise} agent. 
 \begin{table}[]
\begin{tabular}{|l||l|}
\hline
\multicolumn{1}{|c||}{Rank in tournament} & \multicolumn{1}{c|}{Price money (\% of buy-in)} \\ \hhline{=#=}
1 & 31 \% \ \\ \hline
2 & 21.5 \% \ \\ \hline
3 & 16.5 \% \ \\ \hline
4 & 12.5 \% \ \\ \hline
5 & 9 \% \ \\ \hline
6 & 6 \% \ \\ \hline
7 & 3.5 \% \ \\ \hline
\end{tabular}
\centering
\caption{Progressive payout structure for a tournament with 45 players.}
\label{tab:payout_structure}
\end{table} \ \\
Figure \ref{fig:dollar_withoutHOF} shows the profitability of the \textit{Control} agent over 1000 generations. The data was smoothed by a moving average with a subset size of 50. As discussed earlier this graph again shows a non-steady progression of skill over generations. A lot of ups and downs indicate, that good strategies were lost in the phase of evolution. In the end however an overall increase in performance over the span of 1000 generations can be observed. Both fitness evaluations therefore have shown, that the skill of the \textit{Control} agent did progress in this experiment.
\myfig{Results/withoutHOF/dollar_roi.pdf}%% filename in figures folder
  {width=0.7\textwidth,height=0.7\textheight}%% maximum width/height, aspect ratio will be kept
  {Profitability of different playing styles measured by the percentage \textit{ROI}.}%% caption
  {Profitability of different playing styles measured by the percentage \textit{ROI}.}%% optional (short) caption for table of figures
  {fig:roi_withoutHOF}%% label
\myfig{Results/withoutHOF/dollar.pdf}%% filename in figures folder
  {width=1\textwidth,height=1\textheight}%% maximum width/height, aspect ratio will be kept
  {Mean dollar won over 1000 generations.}%% caption
  {Mean dollar won over 1000 generations.}%% optional (short) caption for table of figures
  {fig:dollar_withoutHOF}%% label
  \pagebreak
    %%%
  %%%
  %%%%%%%%%%%%%%%% WITH HOF
\section{Evolution with HOF}
\myfig{Results/withHOF/overallfitness.pdf}%% filename in figures folder
  {width=1\textwidth,height=1\textheight}%% maximum width/height, aspect ratio will be kept
  {Overall fitness of \textit{Control} agent over 1000 generations.}%% caption
  {Overall fitness of \textit{Control} agent over 1000 generations.}%% optional (short) caption for table of figures
  {fig:overallfitness_withHOF}%% label
\myfig{Results/withHOF/progression.pdf}%% filename in figures folder
  {width=1\textwidth,height=1\textheight}%% maximum width/height, aspect ratio will be kept
  {Evolutionary progress for all three individual fitness components over 1000 generations}%% caption
  {Evolutionary progress for all three individual fitness components over 1000 generations}%% optional (short) caption for table of figures
  {fig:progression_withHOF}%% label
\myfig{Results/withHOF/histogram.pdf}%% filename in figures folder
  {width=1\textwidth,height=1\textheight}%% maximum width/height, aspect ratio will be kept
  {Baseline fitness of all playing styles including the \textit{Control} agent.}%% caption
  {Baseline fitness of all playing styles including the \textit{Control} agent.}%% optional (short) caption for table of figures
  {fig:histo_withoutHOF}%% label
\myfig{Results/withHOF/dollar_roi.pdf}%% filename in figures folder
  {width=0.7\textwidth,height=0.7\textheight}%% maximum width/height, aspect ratio will be kept
  {Profitability of different playing styles measured by the percentage \textit{ROI}.}%% caption
  {Profitability of different playing styles measured by the percentage \textit{ROI}.}%% optional (short) caption for table of figures
  {fig:roi_withoutHOF}%% label
\myfig{Results/withHOF/dollar.pdf}%% filename in figures folder
  {width=1\textwidth,height=1\textheight}%% maximum width/height, aspect ratio will be kept
  {Mean dollar won over 1000 generations.}%% caption
  {Mean dollar won over 1000 generations.}%% optional (short) caption for table of figures
  {fig:dollar_withHOF}%% label
%%%%%%%%%%%%%%%% WITH HOF AND OPM
\section{Evolution with HOF \& Opponent Modeling}

%%&TODO Here we compare the playing styles for all free experiments and draw some conclusions out of them. Also we point out the main differences for each method of evolution.
% We also need a nice transition from the HOFAndOPM discussion which leads us here.
 %	-	discuss the results 
\section{Playing Style Evolution \markred{Different title - progression... (weiterentwicklung)}}
Over the curse of 1000 generations the playing style of the neural network agents varied a lot. Figure \ref{fig:player_stats_withoutHOF} shows the evolution of an agent's strategy over all generations. The graph consists of three statistical values used to describe the playing style of a poker player. As described in detail in Subsection \ref{subsec:nnagent} under \textit{Opponent model for all opponents}, the \textit{VPIP}, \textit{PFR} and \textit{AFq} are the three basic statistical values used to  describe tendencies of poker players. \markred{TODO: start describing in little detail how the combination of these three values can be interpreted. Use a reference for that. Then analyze the graph briefly and say how the style of play progressed over time.}
\myfig{Results/withoutHOF/stats.pdf}%% filename in figures folder
  {width=1\textwidth,height=1\textheight}%% maximum width/height, aspect ratio will be kept
  {Player statistics (VPIP, PFR, AFq) over 1000 generations.}%% caption
  {Player statistics (VPIP, PFR, AFq) over 1000 generations}%% optional (short) caption for table of figures
  {fig:player_stats_withoutHOF}%% label
  \myfig{Results/withHOF/stats.pdf}%% filename in figures folder
  {width=1\textwidth,height=1\textheight}%% maximum width/height, aspect ratio will be kept
  {Player statistics (VPIP, PFR, AFq) over 1000 generations.}%% caption
  {Player statistics (VPIP, PFR, AFq) over 1000 generations}%% optional (short) caption for table of figures
  {fig:player_stats_withHOF}%% label
  

%\glsresetall %% all glossary entries should be used in long form (again)
%% vim:foldmethod=expr
%% vim:fde=getline(v\:lnum)=~'^%%%%\ .\\+'?'>1'\:'='
%%% Local Variables:
%%% mode: latex
%%% mode: auto-fill
%%% mode: flyspell
%%% eval: (ispell-change-dictionary "en_US")
%%% TeX-master: "main"
%%% End:

%%%% Time-stamp: <2013-02-25 10:31:01 vk>


\chapter{Conclusion and Future Work }
\label{cha:conclusion}
The goal of the theses was to develop a NL Texas Hold'em poker agent using evolutionary neural networks, that is capable of understanding fundamental concepts of the game to establish a profitable strategy. To tackle problems inherent in evolutionary algorithms a hall of fame was used as a countermeasure in two of three experiments. In another experiment the effect of additional information about opponents on the playing style progression of the population was investigated.\par
The use of a hall of fame for storing successful strategies of previous generations showed a significant increase in steady skill progression. Not only did the population show a much smoother skill progression but also developed a more profitable strategy against static opponents. Adding models of opponent's playing tendencies as input to the neural network, agents developed a more passive strategy compared to the other two experiments. The profitability however did not suffer too much from this different playing style. 
\section{Future Work}
The developed system in this theses gives room to some improvements. Instead of selecting the best agent of a generation by running 1000 tournaments, a better approach would be to run duplicate table tournaments to reduce the variance that comes with the game of poker. Duplicate table tournaments alter the position of all players on the table while dealing the same set of shuffled cards for each tournament. This way the variance can be reduced significantly. Another possible improvement for the system could be achieved by using the hall of fame not only for competition but also let it influence the evolution progress in some way. This could potentially reduce the loss of good strategies over time. Furthermore the structure of the neural network could be improved by altering the number of neurons in the hidden layer. Little effort was made in this thesis to find the optimal number of neurons for the given problem. Further improvement could be achieved by considering a different fitness function or by changing the weights of the fitness components. 

%\glsresetall %% all glossary entries should be used in long form (again)
%% vim:foldmethod=expr
%% vim:fde=getline(v\:lnum)=~'^%%%%\ .\\+'?'>1'\:'='
%%% Local Variables:
%%% mode: latex
%%% mode: auto-fill
%%% mode: flyspell
%%% eval: (ispell-change-dictionary "en_US")
%%% TeX-master: "main"
%%% End:
        

%%%% Time-stamp: <2013-02-25 10:31:01 vk>

\begin{thebibliography}{9}
  \bibitem{master_nuno}
  N. Passos.
  \textit{Poker Learner: Reinforcement Learning Applied to Texas Hold'em Poker},
  Master's thesis, Faculdade de Engenharia da Universidade do Porto, Portugal, 
  2011.
  \href{https://paginas.fe.up.pt/$\sim$ei08029/Master\%20Thesis\%20-\%20Nuno\%20Passos.pdf}{\markblue{https://paginas.fe.up.pt/$\sim$ei08029/Master Thesis - Nuno Passos.pdf}}
  \bibitem{strong_poker}
   J. Schaeffer, D. Billings, L. Pe$\tilde{n}$a, and D. Szafron.
  \textit{Learning to Play Strong Poker}.
  ICML-99, Proceedings of the 16th International Conference on Machine Learning, 
  1999.
  \href{http://poker.cs.ualberta.ca/publications/ICML99.pdf}{\markblue{http://poker.cs.ualberta.ca/publications/ICML99.pdf}}
  \bibitem{pena}
  L. Pe$\tilde{n}$a.
  \textit{Probabilities and simulations in poker.}
  Mather's thesis, Department of Computing Science, University of Alberta,
  1999.
\bibitem{review}
  J. Rubin, and I. Watson.
  \textit{Computer poker: A review.}
   Artificial Intelligence, 175(5-6):958-987,
  2011.
\bibitem{theory_of_poker}
  D. Sklansky.
  \textit{The Theory of Poker.}
   Two Plus Two Publishing,
  2005.
\bibitem{billings_phd}
  D. Billings.
  \textit{Algorithms and Assessment in Computer Poker}
   Ph.D. Dissertation, University of Alberta,
  2006.
\bibitem{poker_dummies}
  R. D. Harroch, and L. Krieger.
  \textit{Poker For Dummies.}
   John Wiley \& Sons, 1st Edition,
  2000.
\bibitem{game_theory}
  M. J. Osborne.
  \textit{A Course in Game Theory.}
   The MIT Press, Cambridge, Massachusetts, London, England,
  2014.
\bibitem{quality_of_bots}
  V. Lisy, and M. Bowling.
  \textit{Equilibrium Approximation Quality of Current No-Limit Poker Bots.}
   arXiv preprint arXiv:1612.07547,
  2017.
\bibitem{selective_sampling}
  D. Billings, L. Pe$\tilde{n}$e, J. Schaeffer, D. Szafron.
  \textit{Using Probabilistic Knowledge and Simulation to Play Poker.}
   In 16th National Conference on Artificial
Intelligence, pp. 697-703,
  1999.
\bibitem{opp_modeling}
  D. Billings, D. Papp, J. Schaeffer, and D. Szafron.
  \textit{Opponent Modeling in Poker.}
   Proc. AAAI-98, Madison, WI, pp. 493-499,
  1998.
\bibitem{opp_master}
  A. Davidson.
  \textit{Opponent modeling in poker: Learning and acting in a hostile and uncertain environment.}
   Master’s thesis, University of Alberta, 
  2002.
\bibitem{ENN_garrett}
G. Nicolai, and R. J. Hilderman.
\textit{No-Limit Texas Hold'em Poker Agents Created with Evolutionary Neural Networks.}
CIG-2009, IEEE Symposium on Computational Intelligence and Games, pp. 125-131, 2009.
\bibitem{evolutionary_methods}
G. Nicolai.
\textit{Evolutionary methods for learning no-limit Texas hold'em poker.} Master's thesis, University of Regina, 2008.
\bibitem{challenge_of_poker}
D. Billings, A. Davidson, J. Schaeffer, and D. Szafron.
\textit{The Challenge of Poker}
Artificial Intelligence 134, pp. 201?240, 2002.
\bibitem{hand_eval}
L. F. Teofilo.
\textit{Estimating the Probability of Winning for Texas Hold'em Poker Agents}.
Proceedings 6th Doctoral Symposium on Informatics Engineering, pp 129-140. Porto, Portugal.
\end{thebibliography}


%\glsresetall %% all glossary entries should be used in long form (again)
%% vim:foldmethod=expr
%% vim:fde=getline(v\:lnum)=~'^%%%%\ .\\+'?'>1'\:'='
%%% Local Variables:
%%% mode: latex
%%% mode: auto-fill
%%% mode: flyspell
%%% eval: (ispell-change-dictionary "en_US")
%%% TeX-master: "main"
%%% End:


%%%% Appendix starts here
\begin{appendices}
\chapter{Glossary of Poker Terms}
This appendix contains definitions and brief explanations of all used poker terms in this bachelor's thesis. A full glossary can also be found on the web, following the link: \href{https://en.wikipedia.org/wiki/Glossary_of_poker_terms}{https://en.wikipedia.org/wiki/Glossary\_of\_poker\_terms}
\begin{itemize}
\item \pokerterm{9-handed}:
A game of poker where 9 players are sitting at the table.
\item \pokerterm{Ante}:
A forced bet in some variants of poker (especially in tournament poker) that every player has to pay in order to receive cards.
\item \pokerterm{Bet}:
A bet describes the amount of chips a player puts into the pot in a betting round when the pot it unopened.
\item \pokerterm{Bet size}:
The amount of chips wagered in a betting action.
\item \pokerterm{Betting action}:
Describes the type of action in a betting situation. For example: Bet, 2-Bet, 3-Bet, 4-bet.
\item \pokerterm{Betting turn}:
\markred{needed?}
\item \pokerterm{Big blind}:
A forced bet that the player two positions to the left of the dealer has to pay before receiving his cards. The Big blind is usually twice the size of the small blind.
\item \pokerterm{Blind structure}:
This is a poker tournament specific term, which describes the temporal structure of blinds. Blinds are usually increased in fixed time intervals.
\item \pokerterm{Blinds}:
Force bets that are split into small blind and big blind. The player immediately to the left of the dealer has to pay the small blind, the player two positions to the left of the dealer has to pay the big blind in order to receive cards.
\item \pokerterm{Buy-In}:
Describes the amount of money payed to enter a tournament and gain chips.
\item \pokerterm{Call}:
Matching a bet made by other players in the current betting round.
\item \pokerterm{Check}:
The action of staying in a hand without committing any money to the pot. This is only possible if the betting round is unopened.
\item \pokerterm{Community board}:
Describes the shared cards on the table as an entity.
\item \pokerterm{Community card}:
Face up dealt cards on the table, shared between all players.
\item \pokerterm{Dealer}:
Either a person on the table dealing cards to players or the position on the table holding the dealer button.
\item \pokerterm{Dealer-button}:
An object marking the seat on the table that receives the last card dealt pre-flop. 
\item \pokerterm{Flop}: The first three cards dealt face up to the community board. It  is followed by the second betting round.
\item \pokerterm{Fold}: 
The action of laying down/ giving up a hand.
\item \pokerterm{Forced bets}:
Bets that have to be payed in order to receive cards pre-flop. This includes the small blind, the big blind and antes.
\item \pokerterm{Hand}:
The cards dealt to a player.
\item \pokerterm{Hand rank}:
The relative strength of a hand at showdown. In Texas Hold'em poker the best 5-card combination defines the hand rank.
\item \pokerterm{Heads-up}:
When only two players are competing in a hand.
\item \pokerterm{Hole cards}:
The two face down cards dealt to a player pre-flop.
\item \pokerterm{Late position}:
Describes the two players sitting on the dealer position and one seat to the right of the dealer (the so-called "Cut-Off" position").
\item \pokerterm{Limit poker}:
A poker variant where there are certain minimum and maximum limits on bet sizes.
\item \pokerterm{Long run}:
Describes a lengthy time period.
\item \pokerterm{No-Limit poker}:
A poker variant where there are no minimum or maximum limits on bet sizes.
\item \pokerterm{Pre-flop}:
A betting round before the flop. In the pre-flop stage of a hand player are being dealt cards and have the chance to bet.
\item \pokerterm{Position}:
The position on the table relative to the dealer seat. Player close to the left of the dealer are in early position and players close to the right of the dealer are in late position.
\item \pokerterm{Pot}:
The accumulated amount of chips wagered by all players combined, that is awarded to the winner of the hand.
\item \pokerterm{Pot odds}:
The ratio between the size of the pot and the amount of chips needed to call.
\item \pokerterm{Probability triple}:
Describes a probability distribution across the three possible betting options \textit{bet/raise, check/call, fold}.
\item \pokerterm{Raise}:
Raising a bet means first matching a bet made by an other player and increase that bet by a certain amount.
\item \pokerterm{River}: The fifth card dealt face up to the community board. It is followed by the fourth and final betting round.
\item \pokerterm{Showdown}:
If more than one player remains after the last betting round (River), the player's hands are exposed and a winner is determined.
\item \pokerterm{Small blind}:
A forced bet that the player immediately to the left of the dealer has to pay before receiving his cards.
\item \pokerterm{Stack}:
The total amount of chips or money a player has currently available to play on the table.
\item \pokerterm{Suited}:
When both hole cards share the same suit
\item \pokerterm{Turn}: The fourth card dealt face up to the community board. It  is followed by the third betting round. 
\end{itemize}
\end{appendices}
%\appendix                       %% closes main document, appendix follows until end; only available in book-classes
%\addpart{Appendix}             %% adding Appendix to tableofcontents

%%\printbibliography              %% remove, if using BibTeX instead of biblatex
%%\bibliographystyle{thebibliography}
%%\bibliography{bibliography}
% \include{further_ressources}  %% this is a suggestion: you have to create this file on demand

%%%% end{document}
\end{document}
%% vim:foldmethod=expr
%% vim:fde=getline(v\:lnum)=~'^%%%%\ .\\+'?'>1'\:'='
%%% Local Variables:
%%% mode: latex
%%% mode: auto-fill
%%% mode: flyspell
%%% eval: (ispell-change-dictionary "en_US")
%%% TeX-master: "main"
%%% End:
