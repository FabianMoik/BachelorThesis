%%%% Time-stamp: <2013-02-25 10:31:01 vk>


\chapter{Introduction}
\label{cha:introduction}

Until recently artificial intelligence (AI) research has focused on creating intelligent algorithms capable of winning against the best human players in a variety of board games such as \textit{chess}, \textit{backgammon} or \textit{go}. In these board games players have perfect information about the entire game state at any given moment, which allows algorithms to brute-force calculate an optimal strategy. Card games, such as \textit{poker} however do not provide perfect information about the game state and therefore provide a much more challenging environment for intelligent algorithms. The game of poker has several interesting characteristics including \textit{imperfect information}, \textit{risk management}, \textit{opponent modeling} and \textit{deception}, that make it an ideal candidate for AI research. Developing a world class artificial poker agent may also help solving open questions in AI research and provide important research benefits. \cite{challenge_of_poker, pena}.\par
 The goal of this thesis is it to develop an artificial poker agent capable of understanding the fundament concepts of \textit{No-Limit Texas Hold'em poker}, to establish a profitable strategy against \textit{static opponents}, by applying \textit{evolutionary algorithms} to an artificial neural network.\par
 Chapter 2 of this thesis gives an introduction to the game of poker, focusing on the \textit{No-Limit} variant of \textit{Texas Hold'em}, and explains the basic rules of the game. Chapter 3 provides an overview of previous work done in the field of computer poker, especially focusing on the creation of artificial poker bots. Chapter 4 discusses the design and implementation of the evolutionary neural network poker bot and explains some key components to achieve a high level poker skill. Moreover the procedure of training and evaluating the neural network agents is explained in detail in this chapter. In Chapter 5 experimental results and interesting observations are discussed. Chapter 6 summarizes conclusions taken from this work and provides possible future work in this field. 

%\glsresetall %% all glossary entries should be used in long form (again)
%% vim:foldmethod=expr
%% vim:fde=getline(v\:lnum)=~'^%%%%\ .\\+'?'>1'\:'='
%%% Local Variables:
%%% mode: latex
%%% mode: auto-fill
%%% mode: flyspell
%%% eval: (ispell-change-dictionary "en_US")
%%% TeX-master: "main"
%%% End:
