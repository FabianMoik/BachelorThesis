%%%% Time-stamp: <2013-02-25 10:31:01 vk>


\chapter{Experimental Results}
\label{cha:results}

%%%% INTRODUCTION

\section{Benchmark Opponents}
% Shortly explain here that we benchmarked against static opponents.
To test the skill of evolved neural network agents, they played in a number of tournaments against predefined static opponents. In a series of tournaments all players in are ranked by the fitness function described in Subsection \markred{x.x}. For all benchmark test the same weight distribution was used to make the results of different evolution methods comparable to each other. While the \textit{average placement} in a tournament might be a strong indicator for the level of skill of a poker player, it certainly should not be used as a benchmark value on its own. In this thesis a combination of three benchmark values was used to assess the skill of an agent. 
% Talk about the weights for each single value and that they were found to be well suiting and were used throughout the testing of all agents

%In a static environment where all players are following a static rule set and do not exploit the weaknesses of their opponents, a \textit{folding strategy} is really strong.
\subsection{Always Fold}
An \textit{Always Fold} agent does exactly what his name suggests, he always folds his two hole cards when it is his turn to bet. The only exception to this rule is when the action allows to check instead of folding. A folding strategy is very effective in a static poker environment, where agents follow static rules and do not exploit weaknesses of their opponents. While a folding strategy can never win against a betting strategy in tournaments, it might frequently reach a low rank because \markred{the blinds are eating it over time and more aggressive strategies bust each other out of the tournament}.
\subsection{Always Call}
\subsection{Always Raise}
\subsection{Random}

\section{Evolution without HOF}
%use the highest ranked agent of the last generation? 
% -> show the fitness over time
%	->	show the overall fitness
%		-	explain why it does not improve with each generation
% ->	show histogram
\section{Evolution with HOF}
\section{Evolution with HOF \& Opponent Modeling}
\section{Result Interpretation with Dollar Won Benchmark}



%\glsresetall %% all glossary entries should be used in long form (again)
%% vim:foldmethod=expr
%% vim:fde=getline(v\:lnum)=~'^%%%%\ .\\+'?'>1'\:'='
%%% Local Variables:
%%% mode: latex
%%% mode: auto-fill
%%% mode: flyspell
%%% eval: (ispell-change-dictionary "en_US")
%%% TeX-master: "main"
%%% End:
