%%%% Time-stamp: <2013-02-25 10:31:01 vk>


\chapter{Poker Basics}
\label{cha:pokerBasics}

\section{What is Poker}
In David Sklansky's \cite{theory_of_poker} poker is being described as a term for a whole family of card games, which can be grouped into different types. In some game variants such as  \textit{Texas Hold'em or Omaha} the player with the highest \pokerterm{hand rank}\TODOrewrite{Rewrite the footnote}\footnote{All poker specific terms mentioned the first time, appear in bold, italics throughout this bachelor's thesis. They are defined and briefly explained in the Appendix.} wins, in others such as \textit{Razz or Badugi} the player with the lowest hand rank wins, and in yet others the \pokerterm{pot} is split between the player with the highest and the player with the lowest hand rank \TODOrewrite{Check if citation is correct}.\par
Among all poker variants there are different \pokerterm{betting} structures. Many poker formats offer \pokerterm{limit} variants of the game and \pokerterm{no-limit} variants. In limit games the \pokerterm{bet size} has an upper and lower limit, whereas in no-limit games the bet size is just restricted by the amount a player has currently left in his \pokerterm{stack} \cite{theory_of_poker}. While bet sizes may not be restricted in no-limit games there are specific rules \markred{throughout} all poker variants that determine the minimum bet size at any given time in the game regardless of the variant of the game itself.
\section{NL Texas Hold'em}

\section{Poker Rules}
\section{Betting}
\section{Hand Rankings}
\section{What Defines a GOOD Poker Player}

%\glsresetall %% all glossary entries should be used in long form (again)
%% vim:foldmethod=expr
%% vim:fde=getline(v\:lnum)=~'^%%%%\ .\\+'?'>1'\:'='
%%% Local Variables:
%%% mode: latex
%%% mode: auto-fill
%%% mode: flyspell
%%% eval: (ispell-change-dictionary "en_US")
%%% TeX-master: "main"
%%% End:
