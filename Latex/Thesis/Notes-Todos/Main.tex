% **************************************************************************************************
% ** SPSC Report and Thesis Template
% **************************************************************************************************
%
% ***** Authors *****
% Daniel Arnitz, Paul Meissner, Stefan Petrik
% Signal Processing and Speech Communication Laboratory (SPSC)
% Graz University of Technology (TU Graz), Austria
%
% ***** Changelog *****
% 0.1   2010-01-25   extracted from report template by Daniel Arnitz (not ready yet)
% 0.2   2010-02-08   added thesis titlepage and modified layout (not ready yet)
% 0.3   2010-02-18   added TUG logo and statutory declaration
% 0.4   2010-02-18   moved the information fields below \input{./base/packages} (encoding...)
% 0.5   2010-03-02   added \ShortTitle to fix problems with long thesis titles
%                    added \ThesisType (makes the template suitable for MSc, BSc, PhD, ... Thesis)
% 0.6   2010-06-05   added pagestyle and pagenumbering after frontmatter, packages has now type
% 0.7   2010-09      \Advisors -> \Assessors, inserted frontmatter for thesis
% 0.8   2010-11      added examples
% 0.9   2011-04      \Twosided now {true,false}, scrbook for thesis (\front-, \main-, \backmatter)
%                    added \SpecialNote for titlepage (funding, etc.), added type "homework"
% 0.10  2011-10-18   fixed two typos in \bibliographystyle{} (bug reported by Michael Tauch)
% 0.11  2011-11-09   fixed/modified preamble (bug reported by Michael Tauch)
% 0.12  2012-07-20   added ./base/opt_macros to deal with optional macros
% 0.13  2012-07-27   added \PaperSize
%
% ***** Todo *****
% - Introduction/Usage
% - explain/show preamble (with \thispagestyle, etc)
% - why doesn't \pagestyle work in preamble while \thispagestyle does? (reported by Markus Fr�hle)
% **************************************************************************************************

% **************************************************************************************************
% basic setup

\newcommand{\DocumentType}{homework} % "thesis" / "report" / "homework"
\newcommand{\DocumentLanguage}{de} % "en" / "de"
\newcommand{\PaperSize}{a4paper} % "a4paper" / "letterpaper"
\newcommand{\Twosided}{false} % "true" / "false"

% **************************************************************************************************
% template setup -- do not change these unless you know what you are doing!
\input{./base/packages_\DocumentType}
\input{./base/layout_\DocumentType}
\input{./base/macros}
% \graphicspath{{./drawings/}{./plots/}{./images/}}
% **************************************************************************************************
% ATTENTION: There is a stylesheet provided for makeindex; set makeindex to -s "./base/index.sty"
% **************************************************************************************************

% uncomment to get watermarks:
% \usepackage[first,bottom,light,draft]{draftcopy}
% \draftcopyName{ENTWURF}{160}


% **************************************************************************************************
% information fields

% general
\newcommand{\DocumentTitle}{General Information}
\newcommand{\DocumentSubtitle}{}
\newcommand{\ShortTitle}{Study Material} % used in headers (keep short!)
\newcommand{\DocumentAuthor}{Fabian Moik}
\newcommand{\DocumentDate}{Graz, \today}
%    for thesis only (will be ignored for reports)
\newcommand{\ThesisType}{Bachelor Thesis}
%\newcommand{\Organizations}{Signal Processing and Speech Communications Laboratory \\ Graz University of Technology, Austria \\[1cm] in co-operation with \\ A Nice Company \\ Cartagena, Spain} % SPSC \\ TUG \\[1cm] in cooperation with \\ A Nice Company
%\newcommand{\Supervisors}{Assoc.Prof. Dipl.-Ing. Dr. Klaus Witrisal \\ Dipl.-Ing. Paul Meissner} % Supervisor 1 \\ Supervisor 2 \\ ...
%\newcommand{\Assessors}{Univ.-Prof. Dipl.-Ing. Dr.techn. Gernot Kubin \\ Assoc.Prof. Dipl.-Ing. Dr. James J. Tobe Defined}
%\newcommand{\SpecialNote}{This work was funded by the Austrian Research Promotion Agency (FFG) under grant 123456.}
%   for report only: revision number
\newcommand{\RevPrefix}{alpha~}
\newcommand{\RevLarge}{1}
\newcommand{\RevSmall}{0}

% confidential? (can of course also be used for other messages/notes)
\newcommand{\ConfidNote}{\today}

\newcommand{\markred}{\textcolor{red}}
\newcommand{\markblue}{\textcolor{blue}}



% **************************************************************************************************
% miscellaneous

% correct bad hyphenation
\hyphenation{}
\usepackage{tabularx}
\usepackage{amsmath}
\usepackage{grffile}
\usepackage{float}
\usepackage{blindtext, graphicx}
\usepackage[labelfont=bf]{caption}
\usepackage{chngcntr}
\usepackage{mathtools}
\usepackage{hyperref}
\usepackage{listings}
\usepackage{color}
\usepackage{scrextend}

\definecolor{dkgreen}{rgb}{0,0.6,0}
\definecolor{gray}{rgb}{0.5,0.5,0.5}
\definecolor{mauve}{rgb}{0.58,0,0.82}
%\lstset{language=[Objective]C, breakindent=40pt, breaklines}

\lstset{ %
  language=C++,                  % the language of the code
  basicstyle=\footnotesize,       % the size of the fonts that are used for the code
  numbers=left,                   % where to put the line-numbers
  numberstyle=\tiny\color{gray},  % the style that is used for the line-numbers
  stepnumber=1,                   % the step between two line-numbers. If it's 1, each line 
                                  % will be numbered
  numbersep=5pt,                  % how far the line-numbers are from the code
  backgroundcolor=\color{white},  % choose the background color. You must add \usepackage{color}
  showspaces=false,               % show spaces adding particular underscores
  showstringspaces=false,         % underline spaces within strings
  showtabs=false,                 % show tabs within strings adding particular underscores
  frame=single,                   % adds a frame around the code
  rulecolor=\color{black},        % if not set, the frame-color may be changed on line-breaks within not-black text (e.g. commens (green here))
  tabsize=4,                      % sets default tabsize to 2 spaces
  captionpos=b,                   % sets the caption-position to bottom
  breaklines=true,                % sets automatic line breaking
  breakatwhitespace=false,        % sets if automatic breaks should only happen at whitespace
  title=\lstname,                 % show the filename of files included with \lstinputlisting;
                                  % also try caption instead of title
  keywordstyle=\color{blue},          % keyword style
  commentstyle=\color{dkgreen},       % comment style
  stringstyle=\color{mauve},         % string literal style
  escapeinside={\%*}{*)},            % if you want to add a comment within your code
  morekeywords={*,...}               % if you want to add more keywords to the set
}
% switches
\newboolean{OptDraftMode}
\newboolean{DisplayContentBoxes}
% \setboolean{OptDraftMode}{true} % optional draft mode for pixel graphics (speed up generation; add \OptDraft to options)
% \setboolean{DisplayContentBoxes}{true} % optional boxes with contents (\ContentBox{Content}{NumPages} can be used as "sticky note" with planned contents)
%   load
\input{./base/opt_macros}

\renewcommand*{\thesection}{\arabic{section}}
\newcommand*{\xchapter}{\setcounter{section}{0}\addchap}
% **************************************************************************************************
% **************************************************************************************************
%
 %**************************************************************************************************
\begin{document}
%%%%%%%%% begin snippet
%% You need to add the package "tabularx".
%% Place the snippet right after \begin{document}

% need tabularx
%\usepackage{tabularx}

\begin{titlepage}
       \begin{center}
             \begin{huge}
				   %% Update assignment number here
                   \textbf{Bachelor Thesis - Poker Simulator}
             \end{huge}
       \end{center}
       \begin{center}
             \begin{large}
		Notes and Todos
             \end{large}
       \end{center}
       \begin{center}
             \begin{large}
                 \textbf{Fabian Moik}
             \end{large}
       \end{center}
\end{titlepage}

%%%%%%%%% end snippet
% **************************************************************************************************
% titlepage
%\input{./base/titlepage_\DocumentType}\emptydoublepage

% for thesis: switch to frontmatter
%\ifthenelse{\equal{\DocumentType}{thesis}}{\pagestyle{empty}\pagenumbering{roman}}{}


% **************************************************************************************************
% **************************************************************************************************
% user-defined part

% FOR THESIS: ADD THE PREAMBLE (ABSTRACT, KURZFASSUNG, ...) HERE (also add an \emptydoublepage in between), e.g.:
%    \input{my-abstract}
%    \emptydoublepage
%    \input{my-kurzfassung}
%    \emptydoublepage
%    ...
% FEEL FREE TO USE \emptypage AND \emptydoublepage TO ADJUST THE LAYOUT
% USE \thispagestyle{empty} for abstract, etc.

% for thesis: statutory declaration
\ifthenelse{\equal{\DocumentType}{thesis}}{\input{./base/declaration}}{}

% TOC
%\emptydoublepage
\tableofcontents

% for thesis: make sure we switch back to standard pagestyles/numbering
\ifthenelse{\equal{\DocumentType}{thesis}}{\emptydoublepage\pagestyle{scrheadings}\pagenumbering{arabic}\mainmatter}

% FOR THESIS: YOU CAN SET THE PAGECOUNTER HERE TO MAKE IT IDENTICAL TO THE PDF PAGE NUMBER
\ifthenelse{\equal{\DocumentType}{thesis}}{\setcounter{page}{7}}{}



%%%%%%%%%%%%%%%%%%%%%%%%%%%%%%%%%%%%%%%%%%%%%%%%%%%%%%%%%%%%%%%%%

% **************************************************************************************************
% mainmatter
\newpage
% %%%%%%%%%%%%%%%%%%%%% 	1	 %%%%%%%%%%%%%%%%%%%%%%%%%
%    \emptydoublepage %FOR THESIS: ALWAYS START CHAPTERS AT RIGHT SIDE
\counterwithin{figure}{section}
\counterwithin{section}{chapter}

%%%%%%%%%%%%%%%%%% TODOs %%%%%%%%%%%%%%%%%%%%
\chapter{TODOs}
%%% Docs
\section{Documentation - Todos}
\begin{itemize}
\item fix the appendix such that it is an own chapter with sections etc.
\item check if chapters and section should be written like this (Poker Basics of Fundamental Poker).
\end{itemize}
%%% Programm
\section{Program/Code - Todos}
\begin{itemize}
\item Get the code to run
\item check what it still missing
\subitem basic opponent models. Remember the VPIP and PFR and rank hands accordingly. \textbf{We can do this on only a single table for testing purposes. To show if it increases the results.}
\item think of a way to show some results
\end{itemize}
%%%%%%%%%%%%%%%%%% OPEN QUESTIONS %%%%%%%%%%%%%%%%%%%%
\chapter{Open Questions}
\section{OPEN:}
\textbf{Where to change the page settings, so that it does not always start on the right and makes blank pages?}\\\\
\textbf{What should be written in cursive and what should be written in bold?}\\
\begin{addmargin}[1em]{2em}% 1em left, 2em right
\markblue{Mild emphasis is cursive and stronger emphasis is bold}\\
\end{addmargin}
\textbf{Where should I put the poker lingua definitions?}
\textbf{What exactly is the appendix used for?}
\section{CLOSE:}
\textbf{what are the basic English writing rules to keep in mind?}\\
\begin{itemize}
\item never user I we or you. Always write in passive voice
\item write out don?t doesn?t it?s -> it is, do not, does not
\item write out e.g (for example) etc?
\item do not use that. Instead use which
\item the first time an acronym is used, write it out in full and place the acronym in parentheses -> Graphical User Interface (GUI)
\end{itemize}
%%%%%%%%%%%%%%%%%% NOTES %%%%%%%%%%%%%%%%%%%%
\chapter{Notes}
\section{THESIS STRUCTURE AND ORGANISATION}
Make the structure like in \textit{Master Thesis - Nuno Passos}. He has a nice structure and we can apply it.
\subsection{BASIC STRUCTURE}
\begin{itemize}
\item Cover page
\item Acknowledgements [OPTIONAL]
\item Abstract (english and german?)
\item Table of Contents (List of Figures, List of Tables [OPTIONAL], List of Abbreviations)
\item (1.) Introduction ($\sim$ 2 pages)
\subitem (1.1) Motivation
\subitem (1.2) Problem Statement
\subitem (1.3) Aim of the Work (Goal)
\subitem (1.4) Structure of the Work
\item (2.) Introduction to Poker (Poker Basics) ($\sim$ 2-3 pages)
\subitem (2.1) What is poker
\subitem (2.2) No-Limit Texas Hold'EM
\subsubitem Poker Rules
\subsubitem Betting
\subsubitem Hand Rankings
\subitem (2.3) \markred{also shortly explain what problems algorithms try to tackle (opponent modeling/player typification, etc...), partial information}
\item \markred{(3.) State-of-the-art/Analysis of existing approaches ($\sim$ 2 pages)} 
\subitem Literature Studies
\subitem Comparison and Summary of Existing Approaches
\item (4.) Suggested Solution/Implementation (Main Part) ($\sim$ 10 - 14 pages)
\subitem (4. 1)\markred{Methodology ($\sim$ 3 - 5 page)}
\subsubitem Used Concepts 
\subsubitem Methods and/or Models (Testbed stuff, NNs, Evolutionary algos)
\subsubitem languages (C++, ...)
\subsubitem \textcolor{red}{design methods, data models, analysis models}
\subsubitem \textcolor{red}{formalisms}
\subitem how did I engineer the tool
\subitem how did I structure the testbed, what are it's capeabilities
\subitem how did I rank/evaluate hands
\subitem how did I setup the NN
\subitem how did I setup the evolutionary algorithm
\subitem what did I observe (results)
\subitem what did I learn from that?
\item (5.) Evaluation/(Results) and Critical Reflection (Conclusion) ($\sim$ 2 - 3 pages)
\item (6.) Summary and Future Work ($\sim$ 1 page)
\item (7.) Appendix (Source Code, Data Models,...)
\item (8.) Bibliography
\end{itemize}
\pagebreak
%%%%%% INTRODUCTION
\subsection{(1.) INTRODUCTION}
Motivate your topic, clearly say what the problem is and why other related techniques/tools don't solve it. clearly state the goals of your work. Discuss the metrics how you can measure success  (e.g. my new algorithm will be faster, people will be able to perform a certain task more quickly and with fewer errors, better user satisfaction, people will gain new insights, I can do something that has not been possible before but is important, Note: you will have to show later that you actually met the goals you claimed
here! briefly explain the process of your work and the methods you used (e.g.
collaborating with domain experts, multiple prototypes, crisply summarize your main contributions at the end of the motivation section
\subsubsection{(1.1) Motivation}
\begin{itemize}
\item poker is a good testbed for AI
\subitem big search space (many variables, many different player types), hidden information unlike chess
\item \markred{why exactly ENN? What do evolutionary algorithms provide that makes them interesting for poker bot?}
\item tournament poker instead of cash game (most research focuses on cash game)
\subitem other fitness function than in cash game (CG: net winning, T: avg. winning (place) etc.)
\end{itemize}
\subsubsection{\markred{(1.2) Problem Statement}}
\begin{itemize}
\item 1.) Environment: 
\subitem mostly not really customizable
\subitem only cash game or other limitations
\subitem not standalone or web dependent
\subitem limited testing opportunities
\subitem less flexible (with my framework, easy change between tournament structure)
\item 2.) Autonomous, profitable poker playing agent in tournament environment:
\subitem  other techniques often focus on cash game and only on heads up play so far, not full table (6 man, 9man).
\subitem other AI techniques successful but ENN has potential (recent scientific work has shown that)
\end{itemize}
\subsubsection{(1.3) Aim of the Work / Goal}
\subsubsection{(1.4) Structure of the Work}
\begin{itemize}
\item briefly explain the upcoming chapters
\item After explaining the structure of the work, shortly mention that a review to what poker is follows (Done here: \cite{review}).
\end{itemize}
\pagebreak
%%%%%%% INTRO TO POKER
\subsection{(2.) INTRODUCTION TO POKER}
\begin{itemize}
\item (2.1) What is poker
\subitem \cite{master_nuno}, \cite{review}
\item (2.2) NL Texas Hold'EM
\subitem \cite{master_nuno}, \cite{pena},  \cite{strong_poker}, \cite{review}
\item (2.3) rules and how the rounds are played out
\subitem \cite{master_nuno}
\subitem explain the 4 rounds of a poker hand
\item (2.4) Betting
\subitem \cite{master_nuno}
\item (2.5) hand rankings (reference to appendix?) \markblue{How to properly do it?}
\subitem \cite{master_nuno}
\subitem maybe just briefly explain the basics of hand ranks and then reference to an indepth list?
\item (2.6) What makes a poker player a good poker player?
\subitem \markred{also shortly explain what problems algorithms try to tackle (opponent modeling/player typification, bluffing, , etc...), partial information} \cite{strong_poker }
\subitem \cite{master_nuno}
\end{itemize}
\pagebreak
%%%%%%%%%% REALTED WORK
\subsection{(3.) RELATED WORK (state-of-the-art/analysis of existing approaches}
The related work section describes existing work that is conceptually similar to what you are working on. It needs to give a good overview of what others have done to overcome the same problem as you face, or conceptually similar problems.\\\\
If you are building, for instance, a network visualization tool for a specific target audience, you should describe state-of-the-art in
network visualization as well as in the target domain.\\\\
\pagebreak
%%%%%%%%%% PRACTICAL PART (Main Part)
\subsection{(4.) PRACTICAL PART (Suggested Solution/Implementation)} 
\subsubsection{(4.1) Methodology}
\begin{itemize}
\item \markred{Something about the Testbed? (HandRanker, HandEvaluator?, Monte Carlo)}
\item Neural Network
\item Evolutionary Algorithm for Training Agents
\end{itemize}
\pagebreak
\subsection{TESTBED}
\ \\
\textbf{Things to mention:}\\
\begin{itemize}
\item Why own testbed: 
\subitem better control, easier to test stuff
\item \textit{Betting System:} faulty implementation (i.e. reraise bet amount when previous all-in < min raise)
\item most of the testbeds only single table and cash game capability. No real tournament (multi-table) simulator out there. 
\end{itemize}
\subsection{FUTURE WORK}
Hand rank algorithm can be improved by taking into account opponent models (i.e Sklansky groups) ->> don't iterate over every possible hand but rather only over likely groups of hands. Also hand potential can be improved this way.
\pagebreak
\begin{thebibliography}{9}
  \bibitem{master_nuno}
  N. Passos.
  \textit{Poker Learner: Reinforcement Learning Applied to Texas Hold'em Poker},
  Master's thesis, Faculdade de Engenharia da Universidade do Porto, Portugal, 
  2011.
  \href{https://paginas.fe.up.pt/~ei08029/Master\%20Thesis\%20-\%20Nuno\%20Passos.pdf}{\markblue{https://paginas.fe.up.pt/$\sim$ei08029/Master Thesis - Nuno Passos.pdf}}
  \bibitem{strong_poker}
   J. Schaeffer, D. Billings, L. Pe$\tilde{n}$a, and D. Szafron.
  \textit{Learning to Play Strong Poker}.
  ICML-99, Proceedings of the 16th International Conference on Machine Learning, 
  1999.
  \href{http://poker.cs.ualberta.ca/publications/ICML99.pdf}{\markblue{http://poker.cs.ualberta.ca/publications/ICML99.pdf}}
  \bibitem{pena}
  L. Pe$\tilde{n}$a.
  \textit{Probabilities and simulations in poker.}
  Mather's thesis, Department of Computing Science, University of Alberta,
  1999.
  \bibitem{review}
  J. Rubin, and I. Watson.
  \textit{Computer poker: A review.}
   Artificial Intelligence, 175(5-6):958-987,
  2011.
  \bibitem{theory_of_poker}
  D. Sklansky.
  \textit{The Theory of Poker}
   Two Plus Two Publishing,
  2005.
  \end{thebibliography}
%\section{Calculating Hand Strength}
%How is EHS calculated: \\
%\href{https://en.wikipedia.org/wiki/Poker\_Effective\_Hand\_Strength\_(EHS)\_algorithm}{\textcolor{blue}{https://en.wikipedia.org/wiki/Poker\_Effective\_Hand\_Strength\_(EHS)\_algorithm}} \\\\
% **************************************************************************************************
% **************************************************************************************************

% place all floats and create label on last page
\FloatBarrier\label{end-of-document}
\end{document}

