% **************************************************************************************************
% **************************************************************************************************
\newsection{Structure of Sections}{intro:sections}

The template provides several pre-defined commands for parts, chapters, sections, subsections, and subsubsections. These commands contain a mandatory argument for the label, and prevent floats (images and tables) to cross part- chapter and section boundaries. Table~\ref{tab:intro:sections:commands} in Subsection~\ref{sec:intro:sections} lists these commands.

\begin{longtable}{l|c|l}
  \textbf{Command} & \textbf{FloatBarrier} & \textbf{Reference As} \\\hline
  \verb|\newpart{Title}{label}| & yes & \verb|\ref{part:label}| \\
  \verb|\newchapter{Title}{label}| & yes & \verb|\ref{chp:label}| \\
  \verb|\newsection{Title}{label}| & yes & \verb|\ref{sec:label}| \\
  \verb|\newsubsection{Title}{label}| & no & \verb|\ref{sec:label}| \\
  \verb|\newsubsubsection{Title}{label}| & no & \verb|\ref{sec:label}| \\
  \caption{Commands to start new parts, chapters, sections, \dots}
  \label{tab:intro:sections:commands}
\end{longtable}





